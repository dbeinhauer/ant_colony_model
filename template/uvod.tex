\chapter*{Úvod}

Kolektivní chování řady druhů hmyzu, mezi než patří například mravenci,
je charakteristické komplexností a vysokou koordinovaností jedinců. Kolonie
mravenců například společně zajišťuje potravu pro celou populaci, buduje hnízdo,
strará se o potomky, či se brání před predátory. Pochopení tohoto chování by
mohlo pomoci zdokonalit řešení řady zdánlivě vzdálených problému, mezi něž
patří například problém obchodního cestujícího (TODO: reference) a řada dalších. 
jedinců, komplexnosti přirozeného prostředí a nepřesnosti měřících zařízeních 
je však exaktní studium chování a organizace těchto kolonií velmi komplikované a 
často značně nepřesné. V důsledku nárustu výpočetního výkonu se v současnosti 
pro studium takto komplexních dějů stále častěji využívají matematické modely. 

V této práci navrhujeme a analyzujeme jednoduchý multiagentní model mravenčí 
kolonie zaměřený na problematiku shánění potravy. Model je založen na komunikaci
mravenců pomocí vypouštění a detekce rozdílné hladiny feromonů v prostředí.
Z velké části je insporován prací 
(TODO: reference https://uwe-repository.worktribe.com/output/980579), jenž studuje
formování transportních drah plísně \emph{Physarum polycephalum}. V analytické části
porovnáváme chování mravenců a jejich úspěšnost při sběru potravy v různě 
strukturovaných prostředích. Dále zkoumáme závislosti počtu jedinců a jejich 
schopnosti dopředu detekovat cílovou destinaci na celkovém množství potravy 
dopravené do hnízda v průběhu simulace. 